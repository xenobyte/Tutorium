\documentclass[german,ignorenonframetext]{beamer} % wir wollen eine Präsentation erstellen

\usepackage[ngerman]{babel} % für deutsche Spracheinstellungen
\usepackage{graphics} % um Bilder einbinden zu können
\usepackage[dvips]{epsfig} % um Bilder zu skalieren
\usepackage[utf8]{inputenc} % wegen deutschen Umlauten (echte Männer nutzen UTF-8!)
\usepackage{hyperref} % um Hyperlinks einfügen zu können

%%%%%%%%%%%%%%%%%%%%%%%%%%%%%%%%%%%%%%%%%%%

%%%%%%% Die folgenden Befehle definieren das Grundlayout, blenden auf der
%%%%%%% Titelseite die HAW-Infos ein und setzen das 
%%%%%%% HAW-Logo in die Ecke
\mode<presentation>{\usetheme{Berkeley}}
\logo{\pgfimage[height=1.5cm]{HAW_wuerfel+}}
\institute[MT -- HAW Hamburg]{HAW Hamburg\\ Fakultät TI, Dept.\ Informatik}

%%%%%%% der folgende Befehl lässt die mit "\pause" verdeckten Teile der Folien 
%%%%%%% transparent erscheinen.  
\setbeamercovered{transparent}  

%%%%%%% die folgende Sequenz blendet mit jeder neuen Section einmal das 
%%%%%%% Inhaltsverzeichnis mit dem Titel "Übersicht" ein und markiert den
%%%%%%% jeweils aktuellen Gliederungspunkt
\AtBeginSection[]{
\begin{frame}<beamer>
\frametitle{Übersicht} 
\tableofcontents[currentsection,currentsubsection]
\end{frame}
}

%%%%%%%%%%%%%%%%%%%%%%%%%%%%%%%%%%%%%%%%%%%

%%%%%%% jeder dieser Titelseiten-Befehle kennt eine in eckige Klammern gesetzte 
%%%%%%% Kurzform, die im Rand benutzt wird
\title[Tutorium]{Tutorium}
\subtitle{TSE / GE1}
\author[Timo Briddigkeit]{Timo Briddigkeit}
\date{\today}

%%%%%%%%%%%%%%%%%%%%%%%%%%%%%%%%%%%%%%%%%%%

\begin{document}

%%%%%% dieser Befehl erzeugt das Deckblatt. 
%%%%%% Mit der Option plain wird das Layout für das Deckblatt abgeschaltet
%\frame[plain]{\titlepage}
\frame{\titlepage}

\begin{frame}
  \frametitle{Übersicht}
  \tableofcontents
\end{frame}

%%%%%%%%%%%%%%%%%%%%%%%%%%%%%%%%%%%%%%%%%%%
\section{Allgemeine Informationen} % die sections sind hier nur zum Strukturieren des Vortrags!!
%% der Name der section taucht nur im Inhaltsverzeichnis und im linken Rand auf.
%%%%%%%%%%%%%%%%%%%%%%%%%%%%%%%%%%%%%%%%%%%
\begin{frame}
\frametitle{Organisatorisches}
In diesem Abschnitt besprechen wir organisatorische Themen und aktuelle Änderungen.
\end{frame}

\begin{frame}
\frametitle{Git-Repository}
Dieses Tutorium wird in einem Git-Repository auf GitHub organisiert.
\begin{center}
\pgfimage[width=0.5\textwidth]{QRGit} \\
\texttt{git clone}
\url{https://github.com/xenobyte/Tutorium.git}
\end{center}
\end{frame}



\begin{frame}
\frametitle{Backlog}
\begin{itemize}
\item Was haben wir beim letzten mal besprochen?
\pause
\item Welche Fragen habt ihr mitgebracht?
\end{itemize}
\end{frame}


\begin{frame}
\frametitle{Eure Anforderungen an dieses Tutorium}
\begin{itemize}
\item mathematische Grundlagen
\pause
\item Klausur bestehen!
\pause
\item Übungsaufgaben rechnen
\pause
\item Tipps von Studenten aus höheren Semestern 
\pause
\item Zeitmanagement
\pause
\item Lernen lernen
\end{itemize}
\end{frame}


%%%%%%%%%%%%%%%%%%%%%%%%%%%%%%%%%%%%%%%%%%%
\section{Grundlagen}
%%%%%%%%%%%%%%%%%%%%%%%%%%%%%%%%%%%%%%%%%%%
\begin{frame}
\frametitle{Was wird in diesem Abschnitt vermittelt?}

Studentische Grundfertigkeiten wie:
\begin{itemize}
\item "Lernen lernen"
\item Zeitmanagement
\item Umgang mit Studienunterlagen
\item \LaTeX
\item MATLAB / GNU Octave
\item LTSpice
\item Editor (Vim / Emacs / Whatever)
\end{itemize}
\end{frame}

%%%%%%%%%%%%%%%%%%%%%%%%%%%%%%%%%%%%%%%%%%%
\subsection{Lernkanäle}
\begin{frame}
\frametitle{Wieviel behaltet ihr wohl?}
\begin{center}
\begin{tabular}{ | l | c | }
\hline
\textbf{Lernkanal} & \textbf{Bewertung} \\ \hline
Lesen & 10\% \\ \hline
Hören & 20\% \\ \hline
Sehen & 30\% \\ \hline
Hören u. Sehen & 50\% \\ \hline
Selbst darüber sprechen & 70\% \\ \hline
Ausprobieren & 90\% \\ \hline
\end{tabular}
\end{center}
\end{frame}

\subsection{Organisation}
\begin{frame}
\frametitle{Zeitmanagement?}
\begin{itemize}
\item Welche Termine habe ich?
\pause
\item Wie organisiere ich Termine?
\pause
\item Wie organisiere ich Freizeit?
\end{itemize}
\end{frame}

\begin{frame}
\frametitle{Dateien organisieren}
Im Studium fallen viele Dateien (Vorlesungsfolien, Bücher, Manuals, usw.) an, die ihr ggf. auf verschiedenen Geräten (Tablet, Laptop, Desktop, Server) synchronisieren wollt.
\begin{itemize}
\pause
\item Dropbox
\pause	
\item owncloud
\pause
\item FTP,SSH, usw. usw.
\end{itemize}
\end{frame}

\begin{frame}
\frametitle{Informationen finden}
Es empfiehlt sich über die Studienunterlagen zu indizieren, um schnell Informationen zu finden. Tools dazu wären z.B.:
\begin{itemize}
\item recoll (Linux / UNIX)
\pause
\item Copernic Desktop Search (Windows)
\pause
\item Spotlight (Mac OS X)
\pause
\item YaCy (P2P Searchengine in Java)
\end{itemize}
\end{frame}


%%%%%%%%%%%%%%%%%%%%%%%%%%%%%%%%%%%%%%%%%%%
\subsection{LaTeX}
\begin{frame}
\frametitle{LaTeX}

Einführung in die Textverarbeitung mit LaTeX\\

\pause
\begin{block}{Hinweis!}
Den LaTeX Code zu diesen Folien und eine Vorlage für Laborprotokolle findet ihr im Git-Repository in dem Verzeichnis \textit{latex}
\end{block}
\end{frame}


%%%%%%%%%%%%%%%%%%%%%%%%%%%%%%%%%%%%%%%%%%%
\subsection{MATLAB}
\begin{frame}
\frametitle{MATLAB}
\begin{itemize}
\item Tool zur Lösung mathematischer Probleme
\pause
\item grafische Darstellung der Ergebnisse
\pause
\item primär für numerische Berechnungen mithilfe von Matrizen ausgelegt
\pause
\item In Hochschulen und der Industrie sehr verbreitet, vor allem für numerische Simulation sowie Datenerfassung, Datenanalyse
\pause
\item Die HAW besitzt eine Hochschullizenz für alle Studenten
\end{itemize}
\end{frame} 

\begin{frame}
\frametitle{MATLAB}
\pgfimage[width=1.0\textwidth]{matlab}
\end{frame} 

%%%%%%%%%%%%%%%%%%%%%%%%%%%%%%%%%%%%%%%%%%%
\section{Elektrotechnik}
%%%%%%%%%%%%%%%%%%%%%%%%%%%%%%%%%%%%%%%%%%%
\begin{frame}
\frametitle{Einige Grundformeln}
\begin{itemize}
\item Spannung?
\item Stromstärke?
\item Widerstand?
\item Leistung?
\end{itemize}
\end{frame} 

\begin{frame}
\frametitle{Formelkreis}
\begin{center}
\pgfimage[width=0.7\textwidth]{formelrad}
\end{center}
\end{frame} 

\begin{frame}
\frametitle{Widerstände}
In Reihe geschaltet:
\begin{eqnarray}
R_{ges} = R_{1} + R_{2} + R_{n}
\end{eqnarray}

Parallel geschaltet:
\begin{eqnarray}
\frac{1}{R_{ges}} = \frac{1}{R_{1}} + \frac{1}{R_{2}} + \frac{1}{R_{n}}
\end{eqnarray}
\end{frame}

\end{document}



%%%%%%%%%%%%%%%%%%%%%%%%%%%%%%%%%%%%%%%%%%%
%%  besondere Gestaltungselemente für Vortragsfolien
%%%%%%%%%%%%%%%%%%%%%%%%%%%%%%%%%%%%%%%%%%%

% Folie
\begin{frame}
\frametitle{Folientitel}
        Text, Bilder, Blöcke
\end{frame}

% neutraler Kasten
\begin{block}{Blocktitel}
        Blocktext
\end{block}

% grüner Kasten
\begin{exampleblock}{Beispielblocktitel}
        Beispielblocktext
\end{exampleblock}

% roter Kasten
\begin{alertblock}{Warnungsblocktitel}
        Warnungsblocktext
\end{alertblock}

% mehrspaltige Folie mit variablen Spaltenbreiten
\begin{columns}
\column{.55\textwidth}
        Text oder Bild
\column{.45\textwidth}
        Text oder Bild
\end{columns}

% sukzessiver Folienaufbau 
\pause

% mehr Befehle für sukzessiven Folienaufbau, 
% der geklammerte Teil erscheint jeweils ab der dritten Teil-Folie
\uncover<3->{}
\only<3->{}
\invisible<1-2>{}

% rote Markierung des geklammerten Teils bei der vierten Teil-Folie
\alert<4>{}

%%%%%%%%%%%%%%%%%%%%%%%%%%%%%%%%%%%%%%%%%%%
