% das Papierformat zuerst
\documentclass[a4paper, 11pt]{article}
% deutsche Silbentrennung
\usepackage[ngerman]{babel}
% wegen deutschen Umlauten
\usepackage[utf8]{inputenc}
% andere pdfs einbinden
\usepackage{pdfpages}
% um bilder einzubinden
\usepackage{graphicx}
% fuer source code
\usepackage{listings}
\usepackage[left=3cm,right=3cm,top=2cm,bottom=2cm]{geometry}
\usepackage{hyperref}

\begin{document}
\title{Aufgabenzettel 2}
\author{Timo Briddigkeit}

%\maketitle
%\newpage

%\tableofcontents
\newpage

\section*{Aufgabe 1}
Eine Kochplatte für $230 V$ nimmt einen Strom von 5,75 A auf.\\
Wie groß ist der Widerstand der Kochplatte?

\section*{Aufgabe 2}
Bei einem Widerstand liegen die Daten 4 k$\Omega$ und 20 mA vor.\\
Wie groß darf die angelegte Spannung im Höchstfall sein?

\section*{Aufgabe 3}
Ein unbelasteter Spannungsteiler für 140 V besteht aus den Widerständen $R_{1}$ = 20 k$\Omega$ und $R_{2}$ = 40 k$\Omega$\\
Wie groß sind die Spannungen $U_{1}$ und $U_{2}$?

\section*{Aufgabe 4}
Drei Uhren mit jeweils 300 $\Omega$ Widerstand sind parallelgeschaltet und an 12 V angeschlossen.\\
Welcher Gesamtstrom fließt in der Anlage?

\section*{Aufgabe 5}
Wie müssen drei Widerstände mit je 6 $\Omega$ geschaltet werden, damit der Gesamtwiderstand 4 $\Omega$ beträgt?

\section*{Aufgabe 6}
Widerstände tragen z.B. die Beschriftung 5 W/100 $\Omega$; 5 W/10 k$\Omega$.\\
An welche höhste Spannung dürfen die Widerstände angeschlossen werden, und wieviel mA fließen dann?

\section*{Aufgabe 7}
Welche Gesamtkapazität ergibt sich, wenn 4 gleiche 40 $\mu$F Kondensatoren
\begin{enumerate}
\item parallel-
\item in Reihe
\end{enumerate}
geschaltet werden?

\section*{Aufgabe 8}
Wieviel Meter Nickeldraht (0,4 $\Omega\ \cdot\ \frac{mm^{2}}{m}$ ) von 0,6 mm Durchmesser werden zur Anfertigung eines Widerstandes von 90 $\Omega$ benötigt?


%%---------------------------------------------------------------------
%% L Ö S U N G E N
%%---------------------------------------------------------------------
%\newpage
%\section*{Lösungen}
%\paragraph{Aufgabe 1}
%\begin{eqnarray}
%R = \frac{U}{I} = \frac{230 V}{5,75 A} = 40 \Omega
%\end{eqnarray}
%
%\paragraph{Aufgabe 2}
%\begin{eqnarray}
%U = R \cdot I = 4000 \Omega \cdot 0,02 A = 80 V
%\end{eqnarray}
%
%\paragraph{Aufgabe 3}
%\begin{eqnarray}
%U_{1} = \frac{140 \cdot 20 000}{60 000} = 46,67 V
%\end{eqnarray}
%
%\begin{eqnarray}
%U_{2} = \frac{140 \cdot 40 000}{60 000} = 93,33 V
%\end{eqnarray}
%
%\paragraph{Aufgabe 4}
%\begin{eqnarray}
%I_{ges} = 3 \cdot I = 3 \cdot 0,04 A = 0,12 A
%\end{eqnarray}
%
%\paragraph{Aufgabe 5}
%\begin{eqnarray}
%(R_{1} + R_{2}) \parallel R_{3}
%\end{eqnarray}
%
%\paragraph{Aufgabe 6}
%\begin{eqnarray}
%U_{1} = \sqrt{P_{1} \cdot R_{1}} = \sqrt{5 W \cdot 100 \Omega} = 22,36 V
%\end{eqnarray}
%
%\begin{eqnarray}
%U_{2} = \sqrt{P_{1} \cdot R_{1}} = \sqrt{5 W \cdot 10000 \Omega} = 223,6 V
%\end{eqnarray}
%
%\begin{eqnarray}
%I_{1} = \frac{U_{1}}{R_{1}} = \frac{22,36 V}{100 \Omega} = 223 mA
%\end{eqnarray}
%
%\begin{eqnarray}
%I_{2} = \frac{U_{2}}{R_{2}} = \frac{223,6 V}{10000 \Omega} = 22 mA
%\end{eqnarray}
%
%\paragraph{Aufgabe 7}
%\begin{eqnarray}
%C_{parallel} = 4 \cdot C = 4 \cdot 40 \mu F = 160 \mu F
%\end{eqnarray}
%
%\begin{eqnarray}
%C_{Reihe} = \frac{C}{4} = \frac{40 \mu F}{4} = 10 \mu F
%\end{eqnarray}

%\paragraph{Aufgabe 8}
%\begin{eqnarray}
%l = \frac{R \cdot q}{p} = \frac{90 \Omega \cdot 0,28 mm^{2}}{0,4 \frac{\Omega \cdot mm^{2}}{m}} = 63m
%\end{eqnarray}
%
%\begin{eqnarray}
%A = \frac{d^{2}}{4} \cdot \pi = \frac{0,6^{2} mm^{2}}{4} \cdot 3,14 = 0,28 mm^{2}
%\end{eqnarray}

\end{document}


